\section{Облачное хранилище лектория}\label{sec:cloud-storage}

%%%%%%%%%%%%%%%%%%%%%%%%%%%%%%%%%%%%%%%%%%%%%%%%%%%%%%%%%
\subsection{Обычный текст}
%%%%%%%%%%%%%%%%%%%%%%%%%%%%%%%%%%%%%%%%%%%%%%%%%%%%%%%%%

Какая-то информация.

%%%%%%%%%%%%%%%%%%%%%%%%%%%%%%%%%%%%%%%%%%%%%%%%%%%%%%%%%
\subsection{Более сложные вещи}
%%%%%%%%%%%%%%%%%%%%%%%%%%%%%%%%%%%%%%%%%%%%%%%%%%%%%%%%%

\hypertarget{lectory-operator-profile}{Данные от профиля оператора}:
\par\hspace{10pt} Логин: my-login
\vspace{-5pt}
\par\hspace{10pt} Пароль: my-password

\vspace{5pt}

\textit{\textbf{Операторы}} загружают видео из аудиторий по пути:
\begin{itemize}
  \item <<Исходники для монтажа>> \ $\ra$ \ <<Raw>> \ $\ra$ \ <<OBS>> \ $\ra$ \ <<\,<год>.<месяц> - <название месяца>\,>> \ $\ra$ \ <<\,<дата>\,>> \ $\ra$ \ <<\,<название предмета и ФИО лектора>\,>>
\end{itemize}

Пути:
\begin{itemize}
  \item <<Исходники для монтажа>> \ $\ra$ \ <<Processed>> \ $\ra$ \ <<OBS>> \ $\ra$ \ <<\,<год>.<месяц> - <название месяца>\,>> \ $\ra$ \ <<\,<дата>\,>> \ $\ra$ \ <<\,<название предмета и ФИО лектора>\,>> --- если занятие записывалось \textbf{на установку в аудитории}.

  \item <<Исходники для монтажа>> \ $\ra$ \ <<Processed>> \ $\ra$ \ <<Camera>> \ $\ra$ \ <<\,<месяц>\,>> \ $\ra$ \ <<\,<дата>\,>> \ $\ra$ \ <<\,<папка вида system\_...>\,>> \ $\ra$ \ <<\,source\_<время лекции>\,>> --- если занятие записывалось \textbf{на переносную камеру}.

        Переносных камер несколько, поэтому папок вида <<system\_...>> может быть несколько --- видео будет в какой-то из них. Рядом с записью будет лежать \textsf{.mp3} файл --- это запись звука на внутренний микрофон камеры. Она может пригодиться, если звук не записался на петличку.
\end{itemize}

Краткая инструкция, как примаунтить диск с аккаунтом \mitem{lectory-operator}:
\begin{enumerate}
  \item Заходим и переключаемся в режим рута:
        \begin{tcolorbox}
          \texttt{sudo -iu root}
        \end{tcolorbox}

  \item Создаём файл с логином/паролем:
        \begin{tcolorbox}
          \texttt{touch .smb\_operator\_credentials}
        \end{tcolorbox}

  \item Прописываем там данные от учётки. В случае учётки оператора:
        \begin{tcolorbox}
          \texttt{username=my-login} \\
          \texttt{password=my-password}
        \end{tcolorbox}

\end{enumerate}

%%%%%%%%%%%%%%%%%%%%%%%%%%%%%%%%%%%%%%%%%%%%%%%%%%%%%%%%%
\subsection{Картинки}
%%%%%%%%%%%%%%%%%%%%%%%%%%%%%%%%%%%%%%%%%%%%%%%%%%%%%%%%%

Картинка с рамкой с максимально возможной шириной, учитывая отступ в (не)нумерованных списках:
\begin{itemize}
  \item Ненумерованный список:
        \begin{center}
          \begin{minipage}[c]{\textwidth - \fboxaddlen - \itemparindent}
            \centering
            \includegraphics[width={\textwidth - \fboxaddlen - \itemparindent},fbox]{Images/watermark.png}
          \end{minipage}
        \end{center}

\end{itemize}

Сжатие картинки по ширине:
\begin{center}
  \begin{minipage}[c]{0.9\textwidth}
    \centering
    \includegraphics[width=0.9\textwidth,fbox]{Images/watermark.png}
  \end{minipage}
\end{center}

%%%%%%%%%%%%%%%%%%%%%%%%%%%%%%%%%%%%%%%%%%%%%%%%%%%%%%%%%
\subsection{Глава, содержащая подглавы}
%%%%%%%%%%%%%%%%%%%%%%%%%%%%%%%%%%%%%%%%%%%%%%%%%%%%%%%%%

%--------------------------------------------------------
\subsubsection{Ссылки}
%--------------------------------------------------------

Важное \noindent\hypertarget{link-to-word}{\textbf{слово}}, на которое будем потом ссылаться.

Чтобы привязать диск, мы будем пользоваться командой \texttt{mount}. \href{https://losst.pro/montirovanie-diska-v-linux}{Тут} можно прочитать про саму команду и её параметры, а \href{https://ubuntuforums.org/showthread.php?t=288534}{тут} --- как настроить подключение по smb протоколу, чтобы диск не удалялся после перезагрузки компьютера (раздел \mitem{Permanent mount}).

\hyperlink{link-to-word}{Ссылаемся} на важное слово внутри документа (обратить внимание, что цвет ссылки другой).

%%%%%%%%%%%%%%%%%%%%%%%%%%%%%%%%%%%%%%%%%%%%%%%%%%%%%%%%%
\subsection{Код}
%%%%%%%%%%%%%%%%%%%%%%%%%%%%%%%%%%%%%%%%%%%%%%%%%%%%%%%%%

\begin{lstlisting}[language=C++]
#include <iostream>

int main() {
    std::cout << "Hello, world\n";
    return 0; // some comment
}
\end{lstlisting}

%%%%%%%%%%%%%%%%%%%%%%%%%%%%%%%%%%%%%%%%%%%%%%%%%%%%%%%%%
\subsection{Проверка переносов ОченьОченьОченьДлинноеНазваниеСПравиламиПереноса}
%%%%%%%%%%%%%%%%%%%%%%%%%%%%%%%%%%%%%%%%%%%%%%%%%%%%%%%%%
